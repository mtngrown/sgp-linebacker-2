table represent aircraft destroyed on
the ground, and only apply to air-
craft in Airfield Alert or Landing/Turn
around boxes on NV player air status
display of airfield attacked. “S”
means airfield destroyed and may not
be used for remainder of game.
No “target of opportunity” bomb-
ing is allowed; a unit must bomb the
target it is targeted for on the US air
status display.

\section*{MOVEMENT}
Movement may only take place during
the movement phase. All units
have a uniform movement allowance
of “one”, enabling them to move
from one map box to another, crossing
the dividing lines or moving along the
lines connecting the map boxes and
the US HOLDING AREA BOX. No
diagonal movement is allowed.

NV air units begin the game in the
airfield alert boxes of the NV air
status display (exception: see setting
up the game). In his movement
phase, the NV player may move all,
any, or none of his aircraft counters
into the “take-off” box, placing the
appropriate echo counters on the
game map, in the map box contain-
ing the airfield of the unit. The echo
counter may then move in the coming
game turns according to the move-
ment rules. The fighter unit/stack of
units must move one step on the
display boxes for each game turn
(movement phase) in the air, and in
the Movement Phase it enters the
Landing/Turn-around box, must
return to its home airfield. If the air
unit cannot return, i.e., the echo
counter is in another map box than
the home airfield, the unit(s) is
removed from play and may not
return till the next day (night, really)
first GT. If the unit ends up in the
South China Sea map box with air-
craft counter in “landing” box, the
aircraft are permanently lost, ceding
victory points to US player.

No airfield may have more than one
echo counter on the game map at
the same time.

No NV units may enter the US
Holding Area Box.

US echo counter have unlimited
staying power (=9 GTs) on the map,
but must return to US holding area
after they have bombed or entered
air-to-air combat (except 8-52 which
only returns after they have bombed).

Units left on the map after the end
of the ninth GT are removed and
placed in the air status display for
the next night (exception: NV units in
South China Sea box as above).

\section*{US PILOT MORALE}
Pilot Morale (PM) is measured on a
6-0 scale where ‘’6’’ is good and
“0’’ is lousy. PM drops one level tor
every flight of B-52’s taking losses
(for this purpose, one echo counter
equals one flight), and raises one
level for every 250 SAMs fired or
seven continuous nights without B-52
missions. PM only affects B-52
missions results.
PM effects:
\begin{verbatim}
6 0r 5 No effect
4 or 3 Subtract ''1'' from bombing
dice rolls
2 or 1 Subtract ''2'' from bombing
dice rolls
0 No B-52 missions allowed
\end{verbatim}

\section*{THE GAME-TURN TRACK}
LINEBACKER II is played in game
turns, each GT consisting of the six
phases making up the Sequence of
Play. Nine GTs make up one night of
real time. No DAYLIGHT GTs are
played. After the ninth GT is com-
pleted, the playing pieces are re-
moved from the game map and set up
on the air status display anew (see
Movement Rules).

\newpage

